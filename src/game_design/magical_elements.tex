\subsection{Magical Elements}
\label{magical-elements}
\unimplemented{}
\draft{}

\begin{figure}
\begin{center}
\begin{tikzpicture}[node distance=6em]
   \tikzstyle{primary} = [draw,rectangle];
   \tikzstyle{secondary} = [draw,diamond];
   \tikzstyle{opposite} = [densely dotted,latex reversed-latex reversed,thin];
   \tikzstyle{component} = [->];
   \node[primary] (WIND) {Wind};
   \node[secondary,below right of = WIND] (SNOW) {Snow};
   \node[primary,below right of = SNOW] (WATER) {Water};
   \node[secondary,below left of = WATER] (PLANT) {Plant};
   \node[primary,below left of = PLANT] (EARTH) {Earth};
   \node[secondary,above left of = EARTH] (METAL) {Metal};
   \node[primary,above left of = METAL] (FIRE) {Fire};
   \node[secondary,above right of = FIRE] (LIGHTNING) {Lightning};

   \draw[component] (WIND) -- (LIGHTNING);
   \draw[component] (WIND) -- (SNOW);

   \draw[component] (WATER) -- (SNOW);
   \draw[component] (WATER) -- (PLANT);

   \draw[component] (EARTH) -- (PLANT);
   \draw[component] (EARTH) -- (METAL);

   \draw[component] (FIRE) -- (METAL);
   \draw[component] (FIRE) -- (LIGHTNING);

   \draw[opposite] (FIRE) -- (WATER);
   \draw[opposite] (WIND) -- (EARTH);
\end{tikzpicture}
\end{center}
\caption{Magical Elements Wheel}
\label{magical-elements-wheel}
\end{figure}

Attacks may have an associated magical element. This element would be one
of four primaries: \textit{Wind}, \textit{Water}, \textit{Earth}, or
\textit{Fire}. A status infliction chance, defended against with some other
attribute, is then rolled to see if the effect occurs. If not, nothing special
happens. Otherwise:
\begin{itemize}
\item
   If the target has no current element afflictions, it is afflicted with a
   level 1 status for that element.
\item
   If the target already has an affliction for that element, this affliction
   increases in level. Cooldown for that element is reset.
\item
   If the target has an affliction for the element facing the one used in the
   attack according to Figure~\ref{magical-elements-wheel} (e.g.~\textit{Wind}
   and \textit{Earth}), the existing affliction's level is lowered and no new
   affliction is added.
\item
   If the target has an affliction of another element, not only does it gain
   a level of affliction for the attack's element, but it also gains an
   affliction corresponding to the element between the two primary elements
   shown in Figure~\ref{magical-elements-wheel} (e.g.~\textit{Wind} plus
   \textit{Water} also adds \textit{Snow}).
\end{itemize}

\begin{itemize}
\item
   Afflictions for composite elements do not have cooldowns: they disappear as
   soon as at least one of the primaries generating it is removed.
\item
   The level of the affliction for a composite element is the maximum between
   the current levels of the afflictions that generate it.
\item
   Having an affliction reach level 0 removes it.
\item
   The mechanics described above prevent having more than three elemental
   afflictions (two primaries and their composite).
\end{itemize}
