\subsection{Damage Types}
\label{damage-types}
There are \textbf{3} damage types: \textit{slashing}, \textit{bludgeoning}, and
\textit{piercing}. Each damage type is
independent, in the sense that an attack of type \textit{A} will only be
affected by the defense of type \textit{A}. This encourages players to
diversify both their attack and defense types so that they are able to exploit
weaknesses in their opponents' defenses all while trying to avoid their
adversaries doing the same. By limiting the types of damage types to 3, there
should be enough depth for players to create interesting team compositions
without the complexity that might lead them to just not bother taking it into
consideration in their planning. It's very much like the
\textit{rock–paper–scissors} or Fire Emblem's \textit{Weapon Triangle}, except
that you are not limited to a single type, and that the type(s) you use for
attack do not have to match the one(s) you use for defense.

Magical elements (Subsection~\ref{magical-elements}) are not damage types.
